\documentclass[11pt]{article}
\usepackage[a4paper,top=2cm,bottom=3cm,left=1.5cm,right=1.5cm]{geometry}
\usepackage{titling}
\usepackage{amsmath}
\usepackage{amssymb}
\usepackage{amsthm}
\usepackage{tikz}
\usepackage{thmtools}
\usepackage[shortlabels]{enumitem}
\usepackage{abstract}
\usepackage{hyperref}

\title{Set Theory}
\date{Spring 2021}

% bold math
\makeatletter
\g@addto@macro\bfseries{\boldmath}
\makeatother

% remove abstract title
\renewcommand{\abstractname}{}
\renewcommand{\absnamepos}{empty}

% style of links
\hypersetup{colorlinks,linkcolor=black}

% set theorem style
\declaretheoremstyle[
  spaceabove=6pt, spacebelow=6pt,
  headfont=\normalfont\bfseries,
  notefont=\normalfont\bfseries,
  bodyfont=\normalfont\upshape,
  postheadspace=0.5em
]{custom}

% set qed symbol
\renewcommand{\qedsymbol}{$\blacksquare$}

% types of theorems
\declaretheorem[style=custom,parent=section]{definition}
\declaretheorem[style=custom,sibling=definition]{example}
\declaretheorem[style=custom,sibling=definition]{theorem}

% use bold fonts to emphasize
\DeclareTextFontCommand{\emph}{\bfseries}

% math operators
\DeclareMathOperator{\pow}{Pow}

\newcommand{\NN}{\mathbb{N}}
\newcommand{\ZZ}{\mathbb{Z}}
\newcommand{\QQ}{\mathbb{Q}}
\newcommand{\RR}{\mathbb{R}}

\begin{document}

% title
\begin{center}
  \LARGE \bfseries \thetitle, \thedate
\end{center}

\begin{abstract}
  This note is taken for the course "Set Theory", which is instructed by Hsueh-I Lu in Spring 2021.
\end{abstract}

\tableofcontents

\section{February 23, 2021}
\subsection{Course Introduction}
\begin{itemize}
  \item Scoring: Midterm exam ($50\%$) and final exam ($50\%$).
  \item References:
  \begin{itemize}
    \item \textsl{Set Theory: A First Course}, by Daniel Cunningham (2016).
    \item \textsl{Set Theory}, by Thomas Jech (2006).
    \item \textsl{Set Theory: An Introduction to Independence Proofs}, by Kenneth Kunen (1983).
    \item \textsl{Elements of Set Theory}, by Herbert Enderton (1977).
  \end{itemize}
\end{itemize}

\subsection{History}
\begin{itemize}
  \item In 1874, it was proved by Georg Cantor that there is no one-to-one correspondence between the set of natural numbers and the set of real numbers.
  Following that proof, the theory of ordinal and cardinal numbers was developed.
  \item In 1908, the first axiomatization of set theory was presented by Ernst Zermelo.
  However, the existence of some infinite sets cannot be proved in this theory.
  \item In 1930, with revisions from Abraham Fraenkel, the axiomatization of Zermelo--Fraenkel set theory was presented, which is currently regarded as the most common foundation for mathematics.
\end{itemize}

\subsection{Why We Need Axiomatic Set Theory}
A \emph{set} is a collection of distinct elements.
One can define a set either by enumerating the elements of a set, or by describing the rules that a set should satisfy.
However, if any properties are allowed to define a set, then one can construct a set which leads to a paradox.
Following are some examples.

\begin{itemize}
  \item Russell's paradox: Suppose that $R = \{S: S \notin S\}$. Does $R \in R$ hold?
  \item Berry's paradox: Suppose that $B$ is the set that contains exactly the smallest positive integer $b$ that is not definable in under sixty letters. Does $b \in B$ hold?
\end{itemize}
%
In order to avoid paradoxes, we need axiomatic set theory as a foundation for mathematics.

\begin{itemize}
  \item The Zermelo--Fraenkel set theory with the axiom of choice included is called \emph{ZFC}.
  \item If the axiom of choice is excluded, then the theory is called \emph{ZF}.
\end{itemize}
%
In ZFC, all sets lie on an infinite hierarchy, which is called the von Neumann universe.

\begin{figure}[h]
  \centering
  \begin{tikzpicture}
    \clip (-7,0) rectangle (7,4.5);
    \draw (0,0) -- (-3,4) (0,0) -- (3,4);
    \draw (-0.45,0.6) -- (0.45,0.6);
    \draw (-0.9,1.2) -- (0.9,1.2);
    \draw (-1.35,1.8) -- (1.35,1.8);
    \draw (-1.8,2.4) -- (1.8,2.4);
    \draw (-2.25,3) -- (2.25,3);
    \draw (-2.7,3.6) -- (2.7,3.6);
    \node[anchor=west] at (0.5,0.3) {$V_0 = \varnothing$};
    \node[anchor=west] at (0.95,0.9) {$V_1 = V_0 \cup \pow(V_0)$};
    \node[anchor=west] at (1.4,1.5) {$V_2 = V_1 \cup \pow(V_1)$};
    \node at (0,2.2) {$\vdots$};
    \node[anchor=west] at (2.3,2.7) {$V_\omega = \bigcup_{\alpha \in \omega} V_\alpha$};
    \node[anchor=west] at (2.75,3.3) {$V_{\omega+1} = V_\omega \cup \pow(V_\omega)$};
    \node at (0,4) {$\vdots$};
  \end{tikzpicture}
  \caption{The von Neumann universe.}
\end{figure}

\end{document}
