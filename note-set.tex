\documentclass[11pt]{article}
\usepackage[a4paper,top=2cm,bottom=3cm,left=1.5cm,right=1.5cm]{geometry}
\usepackage{titling}
\usepackage{amsmath}
\usepackage{amssymb}
\usepackage{amsthm}
\usepackage{tikz}
\usepackage{thmtools}
\usepackage[shortlabels]{enumitem}
\usepackage{abstract}
\usepackage{hyperref}

\title{Set Theory}
\date{Spring 2021}

% bold math
\makeatletter
\g@addto@macro\bfseries{\boldmath}
\makeatother

% remove abstract title
\renewcommand{\abstractname}{}
\renewcommand{\absnamepos}{empty}

% style of links
\hypersetup{colorlinks,linkcolor=black}

% set theorem style
\declaretheoremstyle[
  spaceabove=6pt, spacebelow=6pt,
  headfont=\normalfont\bfseries,
  notefont=\normalfont\bfseries,
  bodyfont=\normalfont\upshape,
  postheadspace=0.5em
]{custom}

% set qed symbol
\renewcommand{\qedsymbol}{$\blacksquare$}

% types of theorems
\declaretheorem[style=custom,parent=section]{definition}
\declaretheorem[style=custom,sibling=definition]{example}
\declaretheorem[style=custom,sibling=definition]{axiom}
\declaretheorem[style=custom,sibling=definition,name=Axiom Schema]{axiomschema}
\declaretheorem[style=custom,sibling=definition]{theorem}
\declaretheorem[style=custom,sibling=definition]{corollary}

% use bold fonts to emphasize
\DeclareTextFontCommand{\emph}{\bfseries}

% math operators
\DeclareMathOperator{\pow}{Pow}
\DeclareMathOperator{\dom}{dom}
\DeclareMathOperator{\ran}{ran}

\newcommand{\id}{\mathrm{id}}

\newcommand{\NN}{\mathbb{N}}
\newcommand{\ZZ}{\mathbb{Z}}
\newcommand{\QQ}{\mathbb{Q}}
\newcommand{\RR}{\mathbb{R}}

\begin{document}

% title
\begin{center}
  \LARGE \bfseries \thetitle, \thedate
\end{center}

\begin{abstract}
  This note is taken for the course "Set Theory", which is instructed by Hsueh-I Lu in Spring 2021.
\end{abstract}

\tableofcontents

\section{February 23, 2021}
\subsection{Course Introduction}
\begin{itemize}
  \item Scoring: Midterm exam ($50\%$) and final exam ($50\%$).
  \item References:
  \begin{itemize}
    \item \textsl{Set Theory: A First Course}, by Daniel Cunningham (2016).
    \item \textsl{Set Theory}, by Thomas Jech (2006).
    \item \textsl{Set Theory: An Introduction to Independence Proofs}, by Kenneth Kunen (1983).
    \item \textsl{Elements of Set Theory}, by Herbert Enderton (1977).
  \end{itemize}
\end{itemize}

\subsection{History}
\begin{itemize}
  \item In 1874, it was proved by Georg Cantor that there is no one-to-one correspondence between the set of natural numbers and the set of real numbers.
  Following that proof, the theory of ordinal and cardinal numbers was developed.
  \item In 1908, the first axiomatization of set theory was presented by Ernst Zermelo.
  However, the existence of some infinite sets cannot be proved in this theory.
  \item In 1930, with revisions from Abraham Fraenkel, the axiomatization of Zermelo--Fraenkel set theory was presented, which is currently regarded as the most common foundation for mathematics.
\end{itemize}

\subsection{Why We Need Axiomatic Set Theory}
A \emph{set} is a collection of distinct elements.
One can define a set either by enumerating the elements of a set, or by describing the rules that a set should satisfy.
However, if any properties are allowed to define a set, then one can construct a set which leads to a paradox.
Following are some examples.

\begin{itemize}
  \item Russell's paradox: Suppose that $R = \{S: S \notin S\}$. Then $R \in R$ if and only if $R \notin R$.
  \item Berry's paradox: Suppose that $B$ is the set that contains exactly the smallest positive integer $b$ that is not definable in under sixty letters. Then $b \in B$ if and only if $b \notin B$.
\end{itemize}
%
In order to avoid paradoxes, we need axiomatic set theory as a foundation for mathematics.

\begin{itemize}
  \item The Zermelo--Fraenkel set theory with the axiom of choice included is called \emph{ZFC}.
  \item If the axiom of choice is excluded, then the theory is called \emph{ZF}.
\end{itemize}
%
In ZFC, all sets lie on an infinite hierarchy, which is called the von Neumann universe.

\begin{figure}[h]
  \centering
  \begin{tikzpicture}
    \clip (-7,0) rectangle (7,4.5);
    \draw (0,0) -- (-3,4) (0,0) -- (3,4);
    \draw (-0.45,0.6) -- (0.45,0.6);
    \draw (-0.9,1.2) -- (0.9,1.2);
    \draw (-1.35,1.8) -- (1.35,1.8);
    \draw (-1.8,2.4) -- (1.8,2.4);
    \draw (-2.25,3) -- (2.25,3);
    \draw (-2.7,3.6) -- (2.7,3.6);
    \node[anchor=west] at (0.5,0.3) {$V_0 = \varnothing$};
    \node[anchor=west] at (0.95,0.9) {$V_1 = V_0 \cup \pow(V_0)$};
    \node[anchor=west] at (1.4,1.5) {$V_2 = V_1 \cup \pow(V_1)$};
    \node at (0,2.2) {$\vdots$};
    \node[anchor=west] at (2.3,2.7) {$V_\omega = \bigcup_{\alpha \in \omega} V_\alpha$};
    \node[anchor=west] at (2.75,3.3) {$V_{\omega+1} = V_\omega \cup \pow(V_\omega)$};
    \node at (0,4) {$\vdots$};
  \end{tikzpicture}
  \caption{The von Neumann universe.}
\end{figure}

\section{March 2, 2021}
\subsection{Logic}
It is sufficient to use the four symbols $\neg$, $\wedge$, $\forall$ and $\in$ when writing first-order logic formulas, and the following abbreviations are adopted for simplicity.
\begin{align*}
  \varphi \vee \psi \quad &\equiv \quad \neg (\neg \varphi \wedge \neg \psi), \\
  \varphi \rightarrow \psi \quad &\equiv \quad \neg \varphi \vee \psi, \\
  \varphi \leftrightarrow \psi \quad &\equiv \quad (\varphi \rightarrow \psi) \wedge (\psi \rightarrow \varphi), \\
  \exists x \varphi \quad &\equiv \quad \neg \forall x \neg \varphi, \\
  x \notin y \quad &\equiv \quad \neg (x \in y), \\
  x = y \quad &\equiv \quad \forall w (w \in x \leftrightarrow w \in y) \wedge \forall z (x \in z \leftrightarrow y \in z), \\
  x \neq y \quad &\equiv \quad \neg (x = y), \\
  x \subseteq y \quad &\equiv \quad \forall w (w \in x \rightarrow w \in y), \\
  x \subset y \quad &\equiv \quad x \subseteq y \wedge x \neq y.
\end{align*}

\subsection{An Empty Set}
\begin{axiom}[Empty Set]
  There exists a set that contains no members.
  \begin{equation*}
    \exists y \forall x (x \notin y).
  \end{equation*}
\end{axiom}

\begin{definition}
  Let $\varnothing$ denote a set that contains no members.
  \begin{equation*}
    \forall x (x \notin \varnothing).
  \end{equation*}
\end{definition}

\subsection{Equality}
\begin{axiom}[Extensionality]
  Two sets are equal if they contain exactly the same elements.
  \begin{equation*}
    \forall x \forall y (\forall w (w \in x \leftrightarrow w \in y) \rightarrow x = y). 
  \end{equation*}
\end{axiom}

\subsection{Unordered Pairs}
\begin{axiom}[Pairing]
  For any two sets $x$ and $y$, there is a set that contains exactly both $x$ and $y$.
  \begin{equation*}
    \forall x \forall y \exists z \forall w (w \in z \leftrightarrow (w = x \vee w = y)).
  \end{equation*}
\end{axiom}

\begin{definition}
  Let $\{x, y\}$ denote the set that contains exactly both $x$ and $y$.
  \begin{equation*}
    \forall x \forall y \forall w (w \in \{x, y\} \leftrightarrow (w = x \vee w = y)).
  \end{equation*}
  Also, let $\{x\} = \{x, x\}$.
  \begin{equation*}
    \forall x \forall w (w \in \{x\} \leftrightarrow w = x).
  \end{equation*}
\end{definition}

\subsection{Unions}
\begin{axiom}[Union]
  For any set $y$, there exists a set that is the union of all members of $y$.
  \begin{equation*}
    \forall y \exists z \forall w (w \in z \leftrightarrow \exists x (w \in x \wedge x \in y)).
  \end{equation*}
\end{axiom}

\begin{definition}
  The union of all members of $y$ is denoted by $\bigcup y$.
  \begin{equation*}
    \forall y \forall w \Bigl(w \in \bigcup y \leftrightarrow \exists x (w \in x \wedge x \in y)\Bigr).
  \end{equation*}
\end{definition}

\begin{definition}
  For any sets $x$ and $y$, we define
  \begin{equation*}
    x \cup y = \bigcup \{x, y\}.
  \end{equation*}
\end{definition}

\subsection{Subsets}
\begin{axiomschema}[Separation]
  For any set $x$ and any property $\varphi$, there exists a set $y$ of $x$ that contains exactly all members of $x$ that satisfy $\varphi$.
  Formally, for any formula $\varphi$ whose free variables are among $v_1, v_2, \dots, v_k, w, x$, there exists an axiom
  \begin{equation*}
    \forall v_1 \forall v_2 \cdots \forall v_k \forall x \exists y \forall w (w \in y \leftrightarrow w \in x \wedge \varphi).
  \end{equation*}
\end{axiomschema}

\begin{definition}
  For any set $x$ and any property $\varphi$, the set that contains members of $x$ that satisfy $\varphi$ is denoted by $\{w \in x: \varphi\}$.
\end{definition}

\begin{definition}
  For any sets $x$ and $y$, we define
  \begin{align*}
    x \cap y &= \{w \in x: w \in y\}, \\
    x \setminus y &= \{w \in x: w \notin y\}.
  \end{align*}
\end{definition}

\begin{definition}
  The intersection of all members of a nonempty set $y$ is denoted by $\bigcap y$.
  \begin{equation*}
    \bigcap y = \Bigl\{w \in \bigcup y: \forall x (x \in y \to w \in x)\Bigr\}.
  \end{equation*}
\end{definition}

\subsection{Power Sets}
\begin{axiom}[Power Set]
  For any set $x$, there exists a set that contains all subsets of $x$.
  \begin{equation*}
    \forall x \exists y \forall z (z \in y \leftrightarrow z \subseteq x).
  \end{equation*}
\end{axiom}

\begin{definition}
  For any set $x$, the set containing all subsets of $x$ is called the \emph{power set} of $x$, denoted by $\pow(x)$.
  \begin{equation*}
    \forall x \forall z (z \in \pow(x) \leftrightarrow z \subseteq x).
  \end{equation*}
\end{definition}

\section{March 9, 2021}
\subsection{Ordered Pairs}
\begin{definition}
  For any sets $x$ and $y$, let $(x, y) = \{\{x\}, \{x, y\}\}$.
\end{definition}

\begin{theorem}
  Let $P = (x, y)$.
  Then
  \begin{align*}
    x &= \bigcap \bigcap P, \\
    y &= \bigcap \Bigl\{w \in \bigcup P: w \notin \bigcap P \vee \bigcap P = \bigcup P\Bigr\}.
  \end{align*}
\end{theorem}
\begin{proof}
  Note that $\bigcap P = \{x\}$ and $\bigcup P = \{x, y\}$.
  It follows that $\bigcap \bigcap P = x$ holds, and we have $\bigcap P = \bigcup P$ if and only if $x = y$.
  Thus,
  \begin{equation*}
    \bigcap \Bigl\{w \in \bigcup P: w \notin \bigcap P \vee \bigcap P = \bigcup P\Bigr\}
    = \bigcap \{w \in \{x, y\}: w \neq x \vee x = y\}
    = \bigcap \{y\}
    = y.
    \qedhere
  \end{equation*}
\end{proof}

\begin{corollary}
  For any sets $u$, $v$, $x$ and $y$, we have $(u, v) = (x, y)$ if and only if $u = x$ and $v = y$.
\end{corollary}
\begin{proof}
  Clearly $u = x$ and $v = y$ imply
  \begin{equation*}
    (u, v)
    = \{\{u\}, \{u, v\}\}
    = \{\{x\}, \{x, y\}\}
    = (x, y).
  \end{equation*}
  If $(u, v) = (x, y)$, then
  \begin{equation*}
    u = \bigcap \bigcap (u, v) = \bigcap \bigcap (x, y) = x
  \end{equation*}
  and
  \begin{align*}
    v
    &= \bigcap \Bigl\{w \in \bigcup (u, v): w \notin \bigcap (u, v) \vee \bigcap (u, v) = \bigcup (u, v)\Bigr\} \\
    &= \bigcap \Bigl\{w \in \bigcup (x, y): w \notin \bigcap (x, y) \vee \bigcap (x, y) = \bigcup (x, y)\Bigr\} \\
    &= y.
    \qedhere
  \end{align*}
\end{proof}

\begin{definition}
  The \emph{Cartesian product} of two sets $X$ and $Y$, denoted by $X \times Y$, is defined by
  \begin{equation*}
    X \times Y = \{P \in \pow(\pow(X \cup Y)): \exists x \exists y (x \in X \wedge y \in Y \wedge (x, y) = P)\}.
  \end{equation*}
\end{definition}

\subsection{Relations}
\begin{definition}
  A \emph{relation} is a set $R$ such that each member of $R$ is an ordered pair.
\end{definition}

\begin{definition}
  For any relation $R$, the \emph{domain} and the \emph{range} of $R$, denoted by $\dom(R)$ and $\ran(R)$, respectively, are defined by
  \begin{align*}
    \dom(R) &= \Bigl\{x \in \bigcup \bigcup R: \exists y ((x, y) \in R) \Bigr\}, \\
    \ran(R) &= \Bigl\{y \in \bigcup \bigcup R: \exists x ((x, y) \in R) \Bigr\}.
  \end{align*}
\end{definition}

\begin{definition}
  For any relation $R$, the \emph{inverse} of $R$, denoted by $R^{-1}$, is defined by
  \begin{equation*}
    R^{-1} = \{P \in \ran(R) \times \dom(R): \exists x \exists y ((x, y) \in R \wedge (y, x) = P)\}.
  \end{equation*}
\end{definition}

\begin{definition}
  For any relations $R$ and $S$, the \emph{composition} of $R$ and $S$, denoted by $S \circ R$, is defined by
  \begin{equation*}
    S \circ R = \{P \in \dom(R) \times \ran(S): \exists x \exists y \exists z ((x, y) \in R \wedge (y, z) \in S \wedge (x, z) = P)\}.
  \end{equation*}
\end{definition}

\begin{definition}
  For any relation $R$ and any set $X$, the \emph{restriction} of $R$ to $X$, denoted by $R|_X$, is defined by
  \begin{equation*}
    R|_X = \{P \in R: \exists x \exists y (x \in X \wedge (x, y) = P\}.
  \end{equation*}
\end{definition}

\begin{definition}
  For any relation $R$ and any set $X$, the \emph{image} of $R$ under $X$, denoted by $R[X]$, is defined by
  \begin{equation*}
    R[X] = \ran(R|_X).
  \end{equation*}
\end{definition}

\subsection{Functions}
\begin{definition}
  A \emph{function} is a relation $f$ such that for any set $x \in \dom(f)$, there exists a unique set $y \in \ran(f)$ such that $(x, y) \in f$.
  If one writes $f: X \to Y$, then $f$ is a function with $\dom(f) = X$ and $\ran(f) \subseteq Y$, and $f$ is called a function from $X$ to $Y$.
  For any function $f: X \to Y$ and any $x \in X$, let $f(x)$ denote the unique member of $Y$ with $(x, f(x)) \in f$.
\end{definition}

\begin{definition}
  Let $X$ and $Y$ be sets.
  \begin{itemize}
    \item An \emph{injection} from $X$ to $Y$ is a function $f: X \to Y$ such that $f^{-1}$ is also a function.
    \item A \emph{surjection} from $X$ to $Y$ is a function $f: X \to Y$ such that $\ran(f) = Y$.
    \item A \emph{bijection} from $X$ to $Y$ is a function $f: X \to Y$ that is both an injection and a surjection from $X$ to $Y$.
  \end{itemize}
\end{definition}

\subsection{Axiom of Choice}
\begin{axiom}[Choice]
  For any relation $R$, there exists a function $f \subseteq R$ with $\dom(f) = \dom(R)$.
\end{axiom}

\begin{theorem}
  Let $f: X \to Y$ be a function with $X \neq \varnothing$.
  Then there exists a function $g: Y \to X$ such that $g \circ f = \id_X$ if and only if $f$ is an injection from $X$ to $Y$.
\end{theorem}
\begin{proof}
  First suppose that $g: Y \to X$ is a function such that $g \circ f = \id_X$.
  For any $y \in Y$, let $x, x' \in X$ such that $(y, x), (y, x') \in f^{-1}$.
  Then we have
  \begin{equation*}
    x = (g \circ f)(x) = g(f(x)) = g(y) = g(f(x')) = (g \circ f)(x') = x'.
  \end{equation*}
  Thus, $f^{-1}$ is a function, implying that $f$ is an injection from $X$ to $Y$.
  \par Now suppose that $f$ is an injection from $X$ to $Y$, and let $x^\star$ be an arbitrary member of $X$.
  Define
  \begin{equation*}
    g = f^{-1} \cup ((Y \setminus \ran(f)) \times \{x^\star\}).
  \end{equation*}
  Note that
  \begin{itemize}
    \item $f^{-1}$ is a function from $\ran(f)$ to $X$, and
    \item $(Y \setminus \ran(f)) \times \{x^\star\}$ is a function from $Y \setminus \ran(f)$ to $X$.
  \end{itemize}
  Thus, $g$ is a function from $Y$ to $X$.
  We have $g \circ f = \id_X$ since
  \begin{equation*}
    (g \circ f)(x) = g(f(x)) = f^{-1}(f(x)) = x
  \end{equation*}
  holds for each $x \in X$.
\end{proof}

\begin{theorem}
  Let $f: X \to Y$ be a function with $X \neq \varnothing$.
  Then there exists a function $g: Y \to X$ such that $f \circ g = \id_Y$ if and only if $f$ is a surjection from $X$ to $Y$.
\end{theorem}
\begin{proof}
  First suppose that $g: Y \to X$ is a function such that $f \circ g = \id_Y$.
  For any $y \in Y$, we have
  \begin{equation*}
    y = (f \circ g)(y) = f(g(y)).
  \end{equation*}
  Thus, $f$ is a surjection from $X$ to $Y$.
  \par Now suppose that $f$ is a surjection from $X$ to $Y$.
  Then $f^{-1} \subseteq Y \times X$ is a relation.
  By the axiom of choice, there exists a function $g: Y \to X$ with $g \subseteq f^{-1}$.
  For any $y \in Y$, since $(y, g(y)) \in g \subseteq f^{-1}$, we have $(g(y), y) \in f$, implying $(f \circ g)(y) = f(g(y)) = y$.
  Thus, $f \circ g = \id_Y$.
\end{proof}

\end{document}
